\chapter{Verification}

\section{Introduction}

     \SLAP\ was verified by comparison of the analytical
results of this code with accelerometer data from the half-scale
Defense High Level Waste (DHLW)
cask test program [3].  The half-scale DHLW cask geometry and
instrumentation locations are shown in Figure B.1.
The cask was dropped
from a height of 30 ft. $\pm$ 1 in. at an initial angle of 10$^\circ
\pm$ 1$^\circ$. such that the narrower end (opposite the closure) of
the cask contacted the unyielding target first.  Comparisons of
experimental and analytical accelerations and velocities (integrated
accelerations) were made at the accelerometer locations A7, A1, and
A3.  A7 is located on the cask bottom end (initial impact end).  A1 is
located slightly toward the closure from the center of gravity.  A3 is
located on the closure (secondary impact) end.

\begin{figure}
\vspace{3.5 in}
\caption{Half-Scale DHLW Cask Model Instrumentation Drawing}
\end{figure}

\section{\SLAP\ Input Development}

      A typical input to \SLAP\ is shown in Table B.1.  The total
mass was determined from the measured weight of the half-scale cask.
The moment of inertia was calculated from the geometry, weight, and
location of the individual cask components.  The moment of inertia
calculation was not experimentally verified.  The location of the cask
center of gravity (length from nose to CG and tail to CG) was also
calculated and verified experimentally.  Agreement between the
calculation and measurement of the center of gravity location was
within .09 in.

\begin{table}
\begin{center}
\caption{Input Parameters to \SLAP\ for the 30 Foot Drop Test of the
DHLW Half-Scale Model at 10$^\circ$}
\makeqnum
\begin{tabular}{||l|c||}
\hline
Geometry: & \\
\quad Nose to CG         &$??25.68$\\
\quad Tail to CG         &$??23.5?$\\
\quad Mass               &$??15.2?$\\
\quad Moment of Inertia  &$8900.??$\\
Nose Spring: & \\
\quad Loading            &$400000.$ at 1.\\
                   &$973000.$ at 3.\\
\quad Unloading          &$8000000.$\\
Tail Spring: &\\
\quad Loading            &$388000.$ at 1.\\
                   &$888000.$ at 2.8\\
\quad Unloading          &$8000000.$\\
\hline
Initial Conditions: & \\
\quad Initial Velocity   &$-527.45$\\
\quad Initial Angle      &$??10.??$\\
\hline
\end{tabular}
\end{center}
\end{table}

    The springs were defined using the results of finite element
analysis and testing of a honeycomb structure similar to
(but much simpler to analyse) that used for
side impact limiters on the DHLW cask.  A ring of aluminum honeycomb
with an inside diameter of 20 in., an outside diameter of 31.75 in. and
an axial length of 10.125 in. was applied to the axial center of a
5500 lb. cylinder. The honeycomb cell structure
was aligned radially, as
in the DHLW impact limiters.  A
sketch of the test structure is shown in Figure B.2.
\begin{figure}
\vspace{3.5 in}
\caption{Honeycomb Crush Test Structure}
\end{figure}
The cylinder was
dropped from 15 and 22 ft.  The cylinder drops were analysed using
DYNA3D [4].  The analysis results for acceleration, final honeycomb
crush (permanent impact limiter displacement), and footprint were in
good agreement with the experimental results. These analysis results
were then used to define the spring behavior of the DHLW impact
limiters required by \SLAP .  The results for the 22-foot test
cylinder drop were
used because the amount of crush matched that expected in the
DHLW impact limiters due to a 30-foot drop.
The acceleration of the
center of gravity of the cylinder predicted by DYNA3D for the 22-foot
drop is shown plotted against the center of gravity displacement in
Figure B.3.
\begin{figure}
\vspace{3.5 in}
\caption{Acceleration versus Displacement for Honeycomb Crush Test
Structure from DYNA3D for a 22-Foot Drop}
\end{figure}
This curve was smoothed, using the Butterworth low pass
filter implemented in GRAFAID [5],
to eliminate the contribution of the high frequency deformation
modes to the acceleration.  In order to apply a Butterworth filter,
the curve must be single valued.
A series of
initial and trailing zeros will also facilitate the filtering
operation.
Thus the analytical curve of Figure B.3
was modified as shown in Figure B.4 by eliminating the unloading
portion of the curve and adding the leading and trailing zeros.
\begin{figure}
\vspace{3.5 in}
\caption{Acceleration versus Displacement for Honeycomb Crush Test
Structure from DYNA3D for a 22-Foot Drop - Filtered}
\end{figure}
After filtering, the acceleration was converted into load
by multiplying by the mass, and then scaled to the length used in the
DHLW impact limiter. This result was approximated with bi-linear
spring definition shown in Fig B.5 and used in the \SLAP\ input
described in Table B.1.
\begin{figure}
\vspace{3.5 in}
\caption{Approximation of Load versus Displacement Behavior of
Honeycomb Impact Limiters on the DHLW Half-Scale Model}
\end{figure}
The unloading modulus was estimated directly
from Figure B.3. The effects of friction were ignored in the {\em
spring} definition.

\section{Comparison of \SLAP\ with Experiment}

The \SLAP\ program writes results at the locations of the nose spring,
the center of gravity, and the tail spring.  For the half-scale DHLW
cask test, these locations do not coincide with the accelerometer
locations. However, because the cask is analysed as a rigid body, the
analytical data may be determined at any arbitrary location by linear
interpolation.  Thus, the \SLAP\ results were interpolated to be
consistent with the accelerometer locations shown in Figure B.1.

     Variation in drop height, within the experimental uncertainty,
has a negligible effect on the cask behavior and thus was ignored.
The initial angle, however, has an appreciable effect on the duration
of the slapdown event along with a minor effect on the magnitudes of
the accelerations and velocities.  Therefore, slapdown analyses were
run with initial angles between 9$^\circ$ and 14$^\circ$.
The values of acceleration and velocity from \SLAP\
for the 13$^\circ$ initial angle are compared to the experimental data
(at the three accelerometer locations A7, A1, and A3) in Figures B.6
- B.11.  The 13$^\circ$ initial angle was chosen for display here
because it
provided the best match for the experimental data, perhaps indicating
some rotation during the 30-foot free drop
of the test.

\begin{figure}
\vspace{3.5 in}
\caption{Comparison of Analytical and Experimental Vertical
Accelerations at Location A7}
\end{figure}

\begin{figure}
\vspace{3.5 in}
\caption{Comparison of Analytical and Experimental Vertical
Accelerations at Location A1}
\end{figure}

\begin{figure}
\vspace{3.5 in}
\caption{Comparison of Analytical and Experimental Vertical
Accelerations at Location A3}
\end{figure}

\begin{figure}
\vspace{3.5 in}
\caption{Comparison of Analytical and Experimental Vertical
Velocities at Location A7}
\end{figure}

\begin{figure}
\vspace{3.5 in}
\caption{Comparison of Analytical and Experimental Vertical
Velocities at Location A1}
\end{figure}

\begin{figure}
\vspace{3.5 in}
\caption{Comparison of Analytical and Experimental Vertical
Velocities at Location A3}
\end{figure}

Because the actual
cask is not perfectly rigid, the accelerometers record some high
frequency response.  The experimental data in the Figures B.6 -
B.11 has been filtered at 500 Hz to allow for easier comparison.  500
Hz was the lowest frequency which did not significantly alter the
rise and fall times and the total pulse width of the data.
Unfortunately, there
is still a significant high frequency component in the accelerometer
response remaining.
Because of this remaining high frequency response, even after
filtering, it is difficult to make accurate quantitative comparisons
of acceleration values. However, when the accelerometer data are
integrated to give velocities, the comparison is remarkably good.
When the coarseness of the spring definition, the neglect of friction,
and the experimental uncertainties are considered, these results
indicate that slapdown events can be analysed with sufficient accuracy
for a great number of purposes including determination of worst
initial angles for testing and the effects of variations in impact
limiter and cask parameters.
